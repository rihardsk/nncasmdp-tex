\documentclass{ludis} % pieejams https://github.com/rihardsk/LU-nosl-guma-darbs---LaTeX

% xelatex
\usepackage{fontspec}
\usepackage{xunicode}
\usepackage{xltxtra}

%\usepackage[utf8]{inputenc}

% languages
\usepackage{fixlatvian}
\usepackage{polyglossia}
\setdefaultlanguage{latvian}
\setotherlanguages{english,russian}

% fonts
%\setmainfont[Mapping=tex-text]{Times New Roman}
%\defaultfontfeatures{Scale=MatchLowercase,Mapping=tex-text}

% bibliography
%\usepackage{csquotes}
\usepackage[
    backend=biber,
    style=numeric-comp,
    sorting=none,
    natbib=true,
    url=false,
    doi=true%,
    %eprint=false
]{biblatex}
\addbibresource{bibliography.bib}

\usepackage[]{hyperref}
\hypersetup{
    colorlinks=false
}
\urlstyle{same}

% toc
\setcounter{secnumdepth}{3}
\setcounter{tocdepth}{3}

%tables
\usepackage{longtable}

%papildus matemātika
\usepackage{mathtools}
\newenvironment{thmenum}
 {\begin{enumerate}[label=\upshape(\arabic*),ref=\thethm(\arabic*)]}
 {\end{enumerate}}
 
%\begin{thmenum}
%\item \label{foo} Foo
%\item \label{bar} Bar
%\item \label{baz} Baz
%\end{thmenum}

%images
\usepackage{graphicx}
\usepackage{float}

%dalīšana kolonnās
\usepackage{multicol}

%saraksti
\usepackage{enumitem}

\fakultate{Datorikas}
\nosaukums{Neironu tīkli un nepārtrauktas darbību telpas Markova izvēles procesi}
\darbaveids{Maģistra kursa}
\autors{Rihards Krišlauks}
\studapl{rk09006}
\vaditajs{Prof., Dr. comp. Valdis Zuters}
%\recenzents{Juris Vīksna profesors Dr.sc.comp.}
\vieta{Rīga}
\gads{2015}

\begin{document}
\maketitle

\begin{abstract-lv}
Abstract-lv
\keywords{atslēgas vārds 1, atslēgas vārds 2}
\end{abstract-lv}
\clearpage

\begin{abstract-en}
Abstract-en
\keywords{Keyword 1, keyword 2}
\end{abstract-en}


\tableofcontents

\specnodala{Apzīmējumu saraksts}
\setlength\LTleft{0pt}
\setlength\LTright{0pt}
\begin{longtable}{| c | p{28em} |}
  \hline
  \textbf{Apzīmējums} & \textbf{Atšifrējums}\\ 
  \endhead

  \hline
  $math$ & Apzīmējuma nosaukums \\
  $math2$ &  Apzīmējuma nosaukums 2\\
  \hline
\end{longtable}

\specnodala{Ievads}
Te ir ievads
\chapter{Markova izvēles procesi}
Markova izvēles procesi (angliski Markov decision processes, turpmāk tekstā - MDP) formalizē un ļauj modelēt izvēles veikšanas procesu apstākļos, kur darbības rezultāts ir atkarīgs tikai no sistēmas pašreizējā stāvokļa, bet ir daļēji nejaušs, t.i., izvēles veicējs procesu kontrolē tikai daļēji.
MDP ļauj modelēt daudz dažādas optimizācijas problēmas, kuru risināšanai tiek lietotas dinamiskās programmēšanas metodes un šajā darbā apskatītā stimulētā mācīšanās.
MDP ir ieviesti \autocite{Bel}.
Mēs lietosim definīciju, kas pieļauj nepārtrauktas stāvokļu un darbību telpas.

\begin{definicija}
Par Markova izvēles procesu, jeb MDP, sauc kortežu $(S, A, T, R)$, kur:
\begin{itemize}
	\item $S \subseteq \mathbb{R}^{D_S}, D_S \in \mathbb{N}$ ir stāvokļu kopa, %TODO iespējams bezgalīga?
	\item $A \subseteq \mathbb{R}^{D_A}, D_A \in \mathbb{N}$ ir darbību kopa, %TODO iespējams bezgalīga?
	\item $T:S \times A \times S \rightarrow [0,1]$ ir pārejas funkcija, kur $T(s, a, s')$ norāda varbūtību, esot stāvoklī $s \in S$ veicot darbību $a \in A$, nonākt stāvoklī $s' \in S$,
	\item $R:S \times A \times S \rightarrow \mathbb{R}$ ir atalgojuma funkcija, $R(s, a, s')$ norāda atalgojumu, kas tiek saņemts, esot stāvoklī $s \in S$ veicot darbību $a \in A$ un pēc tam nonākot stāvoklī $s' \in S$.
\end{itemize}
\end{definicija}
%TODO beigu stāvokļi?

Neformāli to var iedomāties kā procesu, kur darbību veicējs, sauksim viņu par aktieri, var novērot to, kādā %TODO aktieris ir normāls vārds?
stāvoklī sistēma ir pašlaik, un viņam ir pieejama informācija par darbībām, ko ir iespējams veikt.
Aktieris izvēlas darbību un novēro jauno stāvokli, uz kuru pāriet sistēma, kā arī rezultātā saņemto atalgojumu.
Aktiera mērķis ir izvēlēties darbības tā, lai maksimizētu laika gaitā saņemto atalgojumu.
Aktieris iepriekš nezina ne varbūtību, ar kādu tiks veikta viņa izvēlētā pāreja, ne pašu stāvokli, uz kuru pāries sistēma, ne arī atalgojumu, ko saņems.
Šo informāciju viņam ir jāuzkrāj laika gaitā no iepriekšējās pieredzes.

Jāņem vērā, ka uzdevumu sarežģī tieši iepriekš minētā nenoteiktība.
Sākot darbu, aktierim iepriekš nezināmā domēnā, tam trūkst jebkāda informācija par .
Vēlams lai aktieris savas darbības pielāgo tā, lai

Pieminēsim par discount factor un to, ka citādāk var definēt darb. funkc

\chapter{Funkciju aproksimācija}
\chapter{Stimulētā mācīšanās}
\section{apakšnodaļa}
Apakšnodaļas teksts.

\chapter{Diskusija}
\chapter{Secinājumi}
Te ir secinājumi

\printbibliography

\end{document}