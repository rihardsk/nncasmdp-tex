\documentclass{ludis} % pieejams https://github.com/rihardsk/LU-nosl-guma-darbs---LaTeX

% xelatex
\usepackage{fontspec}
\usepackage{xunicode}
\usepackage{xltxtra}

%\usepackage[utf8]{inputenc}

% languages
\usepackage{fixlatvian}
\usepackage{polyglossia}
\setdefaultlanguage{latvian}
\setotherlanguages{english,russian}

% fonts
%\setmainfont[Mapping=tex-text]{Times New Roman}
%\defaultfontfeatures{Scale=MatchLowercase,Mapping=tex-text}

% bibliography
%\usepackage{csquotes}
\usepackage[
    backend=biber,
    style=numeric-comp,
    sorting=none,
    natbib=true,
    url=false,
    doi=true%,
    %eprint=false
]{biblatex}
\addbibresource{bibliography.bib}

\usepackage[]{hyperref}
\hypersetup{
    colorlinks=false
}
\urlstyle{same}

% toc
\setcounter{secnumdepth}{3}
\setcounter{tocdepth}{3}

%tables
\usepackage{longtable}

%papildus matemātika
\usepackage{mathtools}
\newenvironment{thmenum}
 {\begin{enumerate}[label=\upshape(\arabic*),ref=\thethm(\arabic*)]}
 {\end{enumerate}}
 
%\begin{thmenum}
%\item \label{foo} Foo
%\item \label{bar} Bar
%\item \label{baz} Baz
%\end{thmenum}

%images
\usepackage{graphicx}
\usepackage{float}

%dalīšana kolonnās
\usepackage{multicol}

%saraksti
\usepackage{enumitem}

\fakultate{Datorikas}
\nosaukums{Neironu tīkli un nepārtrauktas darbību telpas Markova izvēles procesi}
\darbaveids{Maģistra kursa}
\autors{Rihards Krišlauks}
\studapl{rk09006}
\vaditajs{Prof., Dr. comp. Valdis Zuters}
%\recenzents{Juris Vīksna profesors Dr.sc.comp.}
\vieta{Rīga}
\gads{2015}

\begin{document}
\maketitle

\begin{abstract-lv}
Abstract-lv
\keywords{atslēgas vārds 1, atslēgas vārds 2}
\end{abstract-lv}
\clearpage

\begin{abstract-en}
Abstract-en
\keywords{Keyword 1, keyword 2}
\end{abstract-en}


\tableofcontents

\specnodala{Apzīmējumu saraksts}
\setlength\LTleft{0pt}
\setlength\LTright{0pt}
\begin{longtable}{| c | p{28em} |}
  \hline
  \textbf{Apzīmējums} & \textbf{Atšifrējums}\\ 
  \endhead

  \hline
  $math$ & Apzīmējuma nosaukums \\
  $math2$ &  Apzīmējuma nosaukums 2\\
  \hline
\end{longtable}

\specnodala{Ievads}
Te ir ievads
\chapter{Nodaļa}
\section{apakšnodaļa}
Apakšnodaļas teksts.

\chapter{Secinājumi}
Te ir secinājumi

\printbibliography

\end{document}